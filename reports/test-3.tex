\documentclass[a4paper,11pt]{jsarticle}


% 数式
\usepackage{amsmath,amsfonts,amssymb}
\usepackage{bm}
% 画像
\usepackage[dvipdfmx]{graphicx}
\usepackage{bmpsize}
\usepackage{float}
% 多段組み
\usepackage{multicol}
% リンク
\usepackage{url}
% ハイパーリンク
\usepackage[dvipdfmx]{xcolor}
\usepackage[dvipdfmx]{hyperref}
\usepackage{pxjahyper}
\hypersetup{%
  setpagesize=false,%
  bookmarks=true,%
  bookmarksdepth=tochdepth,%
  bookmarksnumbered=true,%
  colorlinks=true,%
  linkcolor=blue,%
  urlcolor=blue,%
  pdftitle={},%
  pdftitle={},%
  pdfsubject={},%
  pdfauthor={},%
  pdfkeywords={}
}


\begin{document}

\title{段落分けとかのテストをする}
\author{tamos}
\date{\today}
\maketitle


\section{段落1}
段落の一個目だよ
\par
改行できたかな.改行は\verb|"\par"|でできるよ
\par
改行は\verb|"\\"|でもできるよ\\
これは「強制改行」で、改行先で字下げされないよ


enter二回でも改行できるよ
\par
コマンドをそのまま表示するのは\verb|"\verb"|
\par
段落を作るのは\verb|\section{}|
\subsection{サブ段落1}
サブ段落の一個目だよ
\par
サブ段落を作るのは\verb|\subsection{}|
\section{段落2}
段落の二個目だよ.
\par
単純箇条書きをするよ.
\begin{itemize}
\item 角管9mm 3本
\item アルミプレート 10枚
\item NUC 5つ
\end{itemize}
\par
\newpage
\section{段落3}
列挙箇条書きをするよ.
\begin{enumerate}
  \item 朝起きる
  \item ご飯食べる
  \item 歯磨きをする
  \item 着替える
\end{enumerate}
\section{段落4}
\begin{center}
  \includegraphics[width=5cm]{apple.png}\\
  りんごの図
\end{center}
\par
↑これはキャプションをつくれないのでレポートには不向き。
\par
\begin{figure}
  \centering
  \includegraphics[width=10cm]{2025NHKDesktop (1).png}
  \caption{鳳翠}
  \label{鳳と翠}
\end{figure}
ここで、図\ref{鳳と翠}を見てください。
\par
\section{段落5}
数式を書いていくよ.
\par
エネルギー$E$と質量$m$は以下の関係がある
\begin{equation}
  E=mc^{2}
\end{equation}
で関係付けられる。
\par
$c$は光速で、
\begin{equation}
  \label{speed-of-light}
  c=299{.}792{.}458 \, \mathrm{m/s}
\end{equation}
である。
\par
数式番号をつけないときは
\[
  1+1=2
\]
もしくは
\begin{equation*}
  1+2=3
\end{equation*}
でできる.
\par
下付き文字はこう!
\begin{equation}
  e_{i}
\end{equation}
いぇーい
\par
\begin{equation}
  \frac{\pi}{2}=
  \left(\int_{0}^{\infty} \frac{\sin x}{\sqrt{x}} dx \right)^2=
  \sum_{k=0}^{\infty} \frac{(2k)!}{2^{2k}(k!)^2} \frac{1}{2k+1}=
  \prod_{k=1}^{\infty} \frac{4k^2}{4k^2-1}+
  \sum_{j=1}^{\infty} \frac{4j^2 + 2}{3}
\end{equation}
\section{段落6}
\begin{multicols}{3}
ここは3段組み。
\par
でも、3段組にするためにはもっと長い例文が必要なので、意味もないことをここにだらだら書き連ねています。このくらい書けば3段に分かれて組み上げられるでしょうか。こう3段になるんか、思ってたのと違うなぁ、という印象。あははははは
\end{multicols}
\begin{multicols}{2}
ここは2段組み。
\par
でも、2段組にするためにはもっと長い例文が必要なので、意味もないことをここにだらだら書き連ねています。このくらい書けば2段に分かれて組み上げられるでしょうか。こう2段になるんか、思ってたのと違うなぁ、という印象。あははははは
\end{multicols}
\section{段落7}
番号なし箇条書きはこう!
\begin{itemize}
  \item シャンクス
  \item 右腕
  \item ない
\end{itemize}
番号あり箇条書きはこう!
\begin{enumerate}
  \item おまえの
  \item あたまは
  \item わるい
\end{enumerate}
\section{段落8}
\begin{thebibliography}{99}
  \item Leslie Lamport『文書処理システム \LaTeX』(いしい訳, あきの, 1910年)
  \item SOREWA=SOU『溜めれば溜めるほどいい \LaTeX』(こが訳, WADA, 1845年)
\end{thebibliography}
\section{段落9}
\begin{equation}
  A = \begin{pmatrix}
        a_{11} & \ldots & a_{1n} \\
        \vdots & \ddots & \vdots \\
        a_{m1} & \ldots & a_{mn}
      \end{pmatrix}
\end{equation}
\section{段落10}
\begin{tabular}{|r|c|l|} \hline
  \multicolumn{3}{|c|}{WADA}\\ \hline
  あ & い & う\\ \hline
  え & お & か\\
  わだ & WA & SOU\\ \hline
\end{tabular}
\section{段落11}
真空中における光速度の定義(式 \ref{speed-of-light}), \pageref{speed-of-light}ページより、
\par
\section{段落12}
\url{https://texwiki.texjp.org/?LaTeX%E5%85%A5%E9%96%80%2F%E7%9B%B8%E4%BA%92%E5%8F%82%E7%85%A7%E3%81%A8%E3%83%AA%E3%83%B3%E3%82%AF}
\par
\href{http://www.hyuki.com/girl/}{数学ガール}
\par

\end{document}